\section{Выводы}
Выполнив курсовой проект я изучил алгоритм сжатия lzw, который является одним из самых эффективных алгоритмов сжатия, коэффициент сжатия данного алгоритма на большом количестве данных сильно лучше, чем на маленьких данных. Плюсом данного алгоритма является абсолютное отсутствие потерь или искажений при сжатии. Время работы данного алгоритма (зависит еше от того как запрограммировать, здесь используется вариант с использованием unordered\_map в качестве словаря, что дает хорошее преимущество во времени) растет почти линейно при увеличении объема данных, что является большим плюсом(1 млн. символов = 5 млн. микросекунд, 10 млн. сиволов =  50 млн. микросекунд). Также несомненным плюсом данного алгоритма является простота программирования на любом ЯП.
\pagebreak